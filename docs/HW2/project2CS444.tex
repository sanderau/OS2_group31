\documentclass[onecolumn, draftclsnofoot,10pt, compsoc]{IEEEtran}
\usepackage{graphicx}
\usepackage{url}
\usepackage{setspace}

\usepackage{geometry}
\geometry{textheight=9.5in, textwidth=7in}

% 1. Fill in these details
\def \ClassName{		CS 444: Operating Systems II }
\def \TeamNumber{		Group 31}
\def \GroupMemberOne{			Kevin Talik}
\def \GroupMemberTwo{			Austin Sanders}
\def \GroupMemberThree{			Zach Tusing}


			

%%%%%%%%%%%%%%%%%%%%%%%%%%%%%%%%%%%%%%%
\begin{document}
 	\begin{center}
	\huge\bf{ } 
   
    \large\textbf{\textit{ Project 2: Taking an I/O Elevator to Dante's Inferno }}\par
     
    
    
	\small{\bf\textit \ClassName Oregon State University}\par
    \small{\bf{\TeamNumber}}
    
    
    {\bf\textit{ Spring Term, May 6th, 2018} }
    
    
    {\small {\bf Prepared by:} \GroupMemberOne, \GroupMemberTwo, and \GroupMemberThree }
        \end{center}
    %\includegraphics{yoctoLogo.png}
    %\begin{flushright}
    %\small{Figure 69. This is a caption}
    %\end{flushright}
 		\vfill
		\section( Abstract )
		
       \pagebreak
       \section{Introduction }
		 In this project, we were asked test the stability of the I/O elevators of the linux-yocto kernel \url(git.yoctoproject.org/cgit.cgi/linux-yocto/tree/drivers?id=v3.19.2). Instintively, we took turns testing a new elavating device with our body weight until it broke. 

		 This paper will first cover the functionality of an I/O elavator, and a few examples of modern input/output algorithms. 
		 Next, this paper will illuminate the current I/O elavator (FIFO NO-OP) that is in the file system specified above. 
		 With this information, our team will describe our experiences implementing the LOOK I/O Scheduler, and what to expect when swapping out the \textit(virtual) device that reads the input from your \textit(physical) device.
        \end{document}
